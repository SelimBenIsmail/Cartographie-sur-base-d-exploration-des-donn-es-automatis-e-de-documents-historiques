\chapter*{Résumé}
 Ce mémoire de recherche détaille la conception d’un système dont l’objectif est de produire des cartographies de la ville de Douai à l’époque du XIIIe siècle, sur la base de données acquises par une fouille de texte automatisée au travers du registre de rentes de Sire Jehan de France.
 Trois thèmes majeurs sont au centre de ce travail de recherche : l’application des outils de traitement automatique des langues naturelles (TALN) aux documents historiques, l'application d'analyse de réseaux, et la visualisation des données par la cartographie 


