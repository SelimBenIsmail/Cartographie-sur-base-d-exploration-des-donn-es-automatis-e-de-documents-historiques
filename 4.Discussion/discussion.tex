\chapter{Discussion}
\section{Validation ou réfutation de l'hypothèse}
\section{Pistes d'améliorations}
Le mémoire décrit le développement d'un système traduisant un document historique textuel en cartographie. Le processus par lequel passe le document est long  et traverse nombreux champs des technologie de l'information et des humanités numériques. Dans le cadre de ce  mémoire, qui est avant tout un prototype, un \textit{proof of concept}, ces aspects n'ont pu être explorés que dans une profondeur relative, voire de manière très superficielle pour certain. De fait, les pistes d'amélioration du système sont très nombreuse.
%système de table de données
%amélioration de la qualité des données
%clustering des anthroponymes : couplé les méthodes  de calculs de distance  avec une base de données prosopopique afin de bénéficier d'une approche par connaissance. + améliorer l'algorithme
%capture du montant des rentes
%amélioration du code
\section{Conclusion}
