\section{Difficultés rencontrées}
%difficulté document historique
Le travail s'est heurté aux difficultés inhérentes à toutes études de documents historiques.
Ces difficultés peuvent  être partagées en deux catégories : celles intrinsèques au contenu du document et celles issues du support du document. 

Dans la première catégorie, se retrouve les problèmes d'incompréhensions et de confusions liés au vocabulaire, aux glissements sémantiques aux variantes de graphies, ou à la langue utilisée \parencite{piotrowski_natural_2012}.
Notre document, le rentier de Jean de France, est écris en picard médiéval ; une langue relativement proche de l'ancien français. De part ses racines, cette langue possède de grandes ressemblances avec notre français contemporain et permet par conséquent, dans une certaine mesure, une compréhension relative de celle-ci à n'importe quel profane francophone.
Cependant, cette compréhension sommaire ne permet pas d'appréhender les aspects plus subtils de la langue et la signification de certaines formulations reste donc très supposées. 

L'exemple le plus parlant de cette difficulté est le traitement de l'expression << ki fu >> que l'on retrouve abondamment dans le registre. Tout du long de travail, cette expression a été considérée comme le marquage de l'ancien propriétaire de la maison ou du tenement. Mais cela n'est ni plus ni moins, qu'une supposition issue de déductions personnelles et n'est étayée par aucune source. Il se pourrait que l'expression aient plusieurs sens selon sa position dans la phrase : qu'elle désigne l'ancien propriétaire lorsqu'elle est placé devant l'anthroponyme \footnote{ << \textit{Sour le  maison Mgr Sauwalon de Vregelai et sur tout le tenement, \textbf{ki fu Adam le Linnier}} >>} dans le sens << qui fût à >> et que lorsqu'elle est placée après  un anthroponyme, prend le sens de  << qui fût >>, <<feu(e)>>\footnote{ << \textit{Sour 2 maisons \textbf{Daniel, ki fu} clers le le bailliu} >> }. A nouveau, il ne s'agit là que d'interprétations personnelles mais aucune documentation n'a été trouvée afin de répondre à ces questions. Par conséquent, une partie du travail à du s'accomplir dans un spectre d'incertitude avec la réflexion et l'esprit critique pour seuls outils.

La seconde catégories regroupe les difficultés issue du support du document. 
En l'occurrence, notre document est passée d'un format physique en papier en documents texte par plusieurs étapes intermédiaires, chacune entraînant une série d'altérités du document original (erreurs de typographie, d'\textit{OCR}, de transcription , etc...). \footnote{L'historique du document ayant été  largement traité dans le chapitre \textit{Méthodes}, nous n'enterons pas plus dans le détail dans cette partie-ci.}





%surnoms (e.g. le tenement des malades)
%probleme des rentes multiples pour un meme anthroponyme //homonymie
%obtention des carte en opensource : qualité médiocre ou tuiles manquantes à partir d'un certain degré de zoom