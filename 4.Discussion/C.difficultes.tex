\section{Difficultés rencontrées}
%difficulté document historique

\subsection{La complexité du document historique}
Le travail s'est heurté aux difficultés inhérentes à toutes études de documents historiques. Ces difficultés peuvent  être partagées en deux catégories : celles intrinsèques au contenu du document et celles issues du support du document. 

Dans la première catégorie se retrouvent les problèmes d'incompréhensions et de confusions liés au vocabulaire, aux glissements sémantiques aux variantes de graphies, ou à la langue utilisée \parencite{piotrowski_natural_2012}.
Notre document, le rentier de Jean de France, est écrit en picard médiéval ; une langue relativement proche de l'ancien français. De par ses racines, cette langue possède de grandes ressemblances avec notre français contemporain et permet par conséquent, dans une certaine mesure, une compréhension relative de celle-ci à n'importe quel profane francophone.
Cependant, cette compréhension sommaire ne permet pas d'appréhender les aspects plus subtils de la langue et la signification de certaines formulations reste donc très supposée. 

L'exemple le plus parlant de cette difficulté est le traitement de l'expression << ki fu >> que l'on retrouve abondamment dans le registre. Tout du long de travail, cette expression a été considérée comme le marquage de l'ancien propriétaire de la maison ou du tenement. Mais cela n'est ni plus ni moins qu'une supposition issue de déductions personnelles et n'est étayé par aucune source. Il se pourrait que l'expression ait plusieurs sens selon sa position dans la phrase : qu'elle désigne l'ancien propriétaire lorsqu'elle est placée devant l'anthroponyme \footnote{ << \textit{Sour le  maison Mgr Sauwalon de Vregelai et sur tout le tenement, \textbf{ki fu Adam le Linnier}} >>} dans le sens << qui fût à >> et que lorsqu'elle est placée après  un anthroponyme, prend le sens de  << qui fut >>, <<feu(e)>>\footnote{ << \textit{Sour 2 maisons \textbf{Daniel, ki fu} clers le le bailliu} >> }. À nouveau, il ne s'agit là que d'interprétations personnelles, mais aucune documentation n'a été trouvée afin de répondre à ces questions. Par conséquent, une partie du travail a dû s'accomplir dans un spectre d'incertitude avec la réflexion et l'esprit critique pour seuls outils.

La seconde catégorie regroupe les difficultés issues du support du document. 
En l'occurrence, notre document est passé d'un format physique en papier en documents texte par plusieurs étapes intermédiaires, chacune entraînant une série d'altérités du document original (erreurs de typographie, d'\textit{OCR}, de transcription , etc.). \footnote{L'historique du document ayant été  largement traité dans le chapitre \textit{Méthodes}, nous n'enterons pas plus dans le détail dans cette partie-ci.}

\subsection{L'enjeu de la qualité des données}
L'autre principal obstacle auquel le travail s'est confronté, est la dégradation progressive de la qualité des données.
La stratégie adoptée, qui s'appuie sur l'utilisation d'expressions régulières, offre l'avantage de permettre une automatisation des processus sans avoir à recourir aux approches orientées connaissance ou aux approches orientées données. Néanmoins, elle dévoile aussi certains inconvénients : sa forte rigidité face aux exceptions rencontrées et la fragilisation du système que cela implique. Tous les cas possibles doivent être anticipés et préparés, car la moindre irrégularité dans la forme des données traitées entraîne, au mieux, une perte de données, au pire, la corruption d'autres données. Dans le cas de données semi-structurées et non structurées telles que celles extraites du registre de rentes, cela s'avère donc relativement délicat. Bien que le registre soit organisé et présente un certain formalisme dans les énoncés de ces rentes, il s'y trouve encore de nombreuses irrégularités, nous forçant à adapter le code à chacune d'elles. 

La phase de segmentation retourne un ensemble de données relativement correct et suffisant pour poursuivre les phases suivantes.

C'est à partir de là que la qualité des données commence à diminuer, et ce, pour différentes causes.

Dans un premier temps, des erreurs peuvent être incorporées dans le tableau de données lors de la segmentation, celles-ci trouvent parfois leurs origines dans le document source lui-même, comme lorsqu'il s'agit d'erreurs de typographie, d'\textit{OCR}.
Dans d'autres cas, il s'agit des algorithmes qui ne sont pas parfaitement adaptés et qui capturent une chaîne de caractères tronquée ou inadéquate après avoir rencontré une irrégularité dans la morphologie d'une expression, d'un nom ou d'une formule.
Ces erreurs peuvent ensuite se répercuter dans le reste du système. Certaines sont bénignes, mais d'autres vont s'amplifier au fur et à mesure du processus et finir par causer d'importantes pertes de données. 

Ensuite, un autre facteur de la diminution de la  qualité de nos données provient de la troncature d'une partie des données récoltées. Comme il en a déjà été discuté, nous avons estimé le silence moins néfaste que le bruit, dès lors, une partie des données s'est vue être ignorée afin de réduire le risque d'assimiler des données fausses ou conflictuelles. À nouveau, ces pertes d'informations peuvent possiblement en entraîner d'autres, car parmi les données rejetées, certaines auraient potentiellement pu servir à enrichir le tableau de données des sommets. Au terme du processus, même s'il s'agit d'un choix délibéré, nous ne pouvons que déplorer la perte de cette quantité de données.

Il est assez difficile d'évaluer l'impact réel qu'a pu avoir cette diminution progressive de la qualité des données sur les résultats finaux obtenus en sortie du système. Celui-ci n'a pas été l'élément le plus limitant auquel nous avons été confrontés, mais reste,  sans équivoque, un enjeu prépondérant.

\subsection{Problèmes mineurs rencontrés}

Quelques autres problèmes ont été également rencontrés. Ils ont eu des répercussions moindres par rapport à ceux discutés précédemment, mais il convient tout de même de les évoquer.

Tout d'abord, l'observation de différents tableaux et du graphe a permis de mettre en évidence  qu'un même individu peut être débiteur de plusieurs biens arrentés. Un fait qui, bien qu'il puisse paraître évident, a été négligé lors de la conception du système et qui engendre des erreurs ponctuelles dans le graphe (toutes les occurrences d'un même anthroponyme étant modélisées par un unique sommet, alors que ces derniers sont censés correspondre à un emplacement géographique précis).
Une erreur de compréhension du document source et du contexte étudié, qui est finalement  symptomatique des difficultés évoquées plus haut.
Malgré cela, le graphe n'ayant pas été transposé sur le fond de carte et la règle définie pour l'enrichissement du tableau de données des sommets, ces erreurs n'ont pu avoir qu’assez peu d'incidence. 

Un second problème mineur rencontré est celui des ressources cartographiques disponibles. D'une part, une alternative libre d'accès a dû être employée à cause de la récente tarification de l'\textit{API} de \textit{Google Maps}. Malheureusement, le catalogue de cette alternative ne propose pas de fonds de cartes en haute résolution pour la région de Douai. D'autre part, il ne semble pas exister de fonds de cartes << historiques >>. Or, la topographie a drastiquement changé depuis le Moyen Âge.