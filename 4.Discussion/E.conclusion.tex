\section{Conclusion}
Les objectifs fixés au départ ayant été partiellement atteints, il semble compliqué de donner verdict tranché concernant les résultats obtenus. 

Plusieurs aspects doivent être nuancés et prêtent à discussion.
Commençons avec celui de l'automatisation du processus.
Deux interventions de l'agent humain ont été nécessaires : la première consistait en la correction du regroupement des anthroponymes tandis que la seconde consistait à trouver les correspondances entre les éléments topographiques du XIIIe siècle et les éléments topographiques actuels. 

Cela étant, l'automatisation totale d'un tel système, sans plus de ressource, semble utopique. En effet, s'il est concevable qu'un algorithme destiné à regrouper les différentes graphies d'anthroponymes sur performe celui que nous avons utilisé et permette, ainsi, une entière automatisation de cette partie du système, il l'est beaucoup moins que la seconde intervention puisse être automatisée. Celle-ci est déjà d'une certaine complexité lorsqu'elle est exécutée manuellement par l'homme. Elle requiert un travail de recherche approfondi au sein de multiples sources non structurées ainsi qu'une réflexion critique à l'égard de ces dernières.

Un second aspect qu'il convient d'évoquer est le changement de méthodes et d' objectifs intervenu lors de la dernière phase. Fondamentalement, l'objectif a toujours été d’inscrire sur la carte les demeures des différents individus cités dans le registre, et en ce sens, les cartes produites rentrent dans ce cadre. Cette première perspective, qui suggérait de transposer le graphe sur le fond de carte, en cherchant à profiter du fait que les maisons sont décrites par leurs attenances dans le registre, afin de reconstituer les allées des rues, est finalement un usage assez peu conventionnel de l'analyse de réseau. Par conséquent, le projet s'est heurté à une contrainte technologique où les outils utilisés n'étaient pas en mesure de répondre aux besoins de cette approche. Au lieu de cela, le graphe a été utilisé pour déterminer les informations topographiques de certains sommets à partir de celles des sommets voisins. 

Cependant, si l'objectif initial n'est pas totalement atteint, il en demeure que les cartographies produites ne sont pas dénuées d'intérêts. Elles permettent une approche visuelle des données textuelles contenues dans le registre de rentes. De plus, les différents graphes et tableaux de données créés pour parvenir à ces cartographies ont eux aussi leur part d'intérêt. Tandis que les tableaux sont une énumération des données prélevées, les graphes, eux, matérialisent les relations qui les unissent. Il en ressort une réserve d'informations structurées, exploitables en vue d'étudier différents aspects de la réalité décrite par le registre de rentes de Jean de France ou d'émettre de nouvelles hypothèses. Pour donner quelques exemples : le tableau de données principal peut être utilisé afin d'étudier les montants et différentes espèces de rentes en fonction des zones topographiques, tandis que le tableau de correspondances des connétablies  peut être réutilisé dans d'autres études de documents contemporains à Jean de France.
Nous pouvons supposer également que l'analyse de la densité des sous-graphes permettrait d'identifier les centres névralgiques de la ville et que l'observation des connexions entre sommets de différentes connétablies puisse servir à retrouver la localisation de connétablies, qui aujourd'hui, ont disparu (à l'image de celles longeant les fossets ou la rue fait-en-paille).

En l'état actuel des choses, la justesse ainsi que la précision des cartes produites sont insuffisantes pour qu'elles puissent être utilisées seules ou comme références.
Mais il ne s'agit là que d'un prototype, presque d'une étude de faisabilité. Les résultats, bien d'imparfaits, me semblent déjà encourageants. Selon moi, si les différents éléments du système sont améliorés, cette approche pourrait être adaptée à d'autres documents historiques de même nature, et devenir un outil pertinent pour visualiser ou agréger l'information contenue dans de gros corpus de documents. 

Ceci démontre une nouvelle fois que les méthodes d'analyses quantitatives, bien que parfois complexes à mettre en place,  apportent une réelle complémentarité à l'étude traditionnelle des sciences humaines.

