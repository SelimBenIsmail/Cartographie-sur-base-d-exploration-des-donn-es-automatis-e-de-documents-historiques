\section{Validation ou réfutation de l'hypothèse}
\subsection{Analyse de la qualité des données}
Très certainement, le plus gros obstacle auquel le travail s'est heurté est la dégradation progressive de la qualité des données à mesure qu'on les traitait.
La stratégie qui a été adoptée, basée sur l'utilisation d'expressions régulières offre l'avantage de permettre une automatisation des processus sans avoir à recourir aux approches orientées connaissance ou aux approches orientées données. Néanmoins, elle dévoile aussi certain inconvénients : sa forte rigidité face aux exceptions rencontrées et la fragilisation du système que cela implique. Tous les cas possibles doivent être anticipés et préparés car la moindre irrégularité dans la forme des données traitées entraîne, au mieux, une perte de données, au pire, la corruption d'autres données ou un risque de blocage du système.

Dans le cas de données semi-structurées et non-structurées telles que celles extrait du registre de rentes cela s'avère donc relativement délicat. Bien que le registre soit structuré et présente un certain formalise dans l'énoncé de ces rentes, on y retrouve tout de même de nombreuses d'irrégularités qui nous force à adapter le code à chacune d'elle. 

La phase de segmentation retourne un ensemble de données relativement correcte et suffisant pour mener un certain nombre d'analyses comme, par exemple, la distribution des rentes par escroete ou connétablie. 
% ---- %
La qualité des données diminue pour plusieurs cause.

Dans un premier temps, on retrouve des erreurs entrées dans le tableau de données lors de la segmentation, trouvant parfois leur origine dans le document source lui-même (e.g. erreurs de typographie non corrigées ou irrégularités dans les expressions et formules utilisées), qui se répercute dans l'ensemble dus système. Celles-ci peuvent s'amplifier au fur et à mesure des étapes du processus et causer d'importante pertes de données. 

% cause 2:  la suppression (tronquage) de données pour éviter les données bruité ou corrompues 


% finir%



