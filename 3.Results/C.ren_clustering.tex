
\section{Reconnaissance des entités nommées}
La reconnaissance des entités nommées s'effectue au sein des sections correspondant aux textes des rentes qui sont stockées dans le tableau de données principal. Chacune de ces sections est fouillée à l'aide d'une expression régulière afin de détecter les caractéristiques morpho-syntaxiques correspondant aux anthroponymes du  nord  de la France au Moyen-Âge.

L'expression régulière dont il est question nous est donnée par les travaux de S. De Valeriola. \\
 $/[:upper:][:lower:]+ (((l[aei']s?|d[euo']l?u?|au?)?)\{0,2\} ?[:upper:][:lower:]+(-[:upper:][:lower:]+)?)\{1,3\}/$\footfullcite[p.10]{de_valeriola_lordinateur_2021}

Celle-ci peut être décomposée en trois sous-expressions, correspondant chacune à une partie de l'anthroponyme \parencite{de_valeriola_lordinateur_2021}.
\begin{itemize}
    \item La première, $/[:upper:][:lower:]+ /$, capture le nom de baptême de la personne.
    \item La seconde , $/(((l[aei']s?|d[euo']l?u?|au?)?)\{0,2\} ?/ $ permet, quant à elle, de capturer les éventuelles particules de patronyme ( par exemple : <<de>>, <<dou>>, <<le>>, <<l'>> , <<de le>>, etc.).
    \item Finalement,  la dernière partie , $ /[:upper:][:lower:]+(-[:upper:][:lower:]+)?)\{1,3\}/ $  sert à capturer le reste du patronyme, sans les éventuelles particules.
\end{itemize}

Grâce à cette RegEx, la fonction \textit{str\_extract\_all()} du package \textit{StringR} nous retourne 799 formes distinctes d'anthroponyme. 


\subsection{Regroupement des anthroponymes}
Comme  il en a été discuté dans le chapitre \textit{Méthodes}, le choix de la méthode et de la distance de seuil pour le regroupement des anthroponymes est une étape aussi critique que délicate.
Afin de choisir la stratégie la plus adéquate, des tests on été effectués sur les 100 premiers anthroponymes, noms de baptême et patronymes recensé à travers le registre.

A travers ces tests, il a été essayé d'observer la dispersion des couples,corrects et incorrects, déterminés par l'algorithme en fonction de la distance, l'évolution du taux de bruit en fonction de la distance, et l'impact, positif ou négatif, d'une modification de pondération des opérations\footnote{Cette modification de pondération s'ajoute à celles évoquées dans le chapitre \textit{Méthodes} provenant de l'article de \fullcite[p.14]{de_valeriola_lordinateur_2021}} dans le calcul de la distance.
Cette modification de pondération consiste à passer le poids des opérations de substitution de 1,0 à 1,25. 
L'idée derrière cette modification est d'alourdir significativement le score lorsque plusieurs opérations de substitution son utilisées. 
Étant donné que la plupart des variants morphologiques d'un anthroponymes consistent en l'ajout ou la suppression d'une lettre ou d'une particule, les opérations de substitution sont relativement rare et ont un impact plus lourd que les autres opération sur la modification de la chaîne caractères.

La distance Damerau-Levensthein  entre chaque éléments testés a été calculée et tous les couples d'éléments dont la distance était égale ou inférieure à la distance de seuil (changeante en fonction de la nature des éléments testés) ont été retenus dans un tableau de données. Après quoi, chaque couple a été manuellement vérifié et annoté <<correct>> ou <<incorrect>> par l'agent humain.

Six configurations ont donc été analysées :
\begin{itemize}
    \item sur l'anthroponymes complet (nom de baptême + patronyme), considéré comme une seule chaîne de caractères, avec une distance de seuil à 3.
    \item Sur l'anthroponymes complet, avec une distance de seuil à 4, et avec avec la modification de la pondération
    \item Sur le nom de baptême uniquement, avec une distance de seuil à 2.
    \item Sur le nom de baptême uniquement, avec une distance de seuil à 2, et avec la modification de pondération.
    \item Sur le patronyme uniquement, avec une distance de seuil à 3.
    \item Sur le patronyme,avec une distance de seuil à 3, et avec la modification de pondération.
\end{itemize}

