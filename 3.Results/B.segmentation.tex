\section{Segmentation du texte}
La phase de segmentation  est la première étape d'extraction des données.
L'algorithme est développé sur les vingts premières pages du registre. Lorsque celui-atteint une précision suffisante sur ce set de données réduit, il est alors testé et évalué sur l'ensemble du registre.

Lorsque le document source est importé dans l'environnement de développement en format texte (\textit{.txt}), ce dernier va découper le contenu textuel en petites chaînes de caractères au niveau des espaces et stocker celles-ci dans un vecteur de type caractère. Ceci présente l'avantage que chaque mot ou marqueur de structure peut être désigné par sa position à l'intérieur du vecteur. Par conséquent, capturer une section spécifique du texte, nécessite uniquement de pouvoir déterminer les positions du début et de fin de celle-ci au sein du vecteur.  

Le document source est donc transformé en un vecteur \textit{T} de 15390 éléments.
Quant aux procédés de détection et de capture des différentes parties du document source, ils peuvent être décrits par les expressions régulières et les formules contenues dans le tableau \ref{regexSeg}.
\vspace{0,5cm}
\renewcommand{\arraystretch} {1.5}
\begin{table}[ht]
    \centering
    \begin{tabular}{|l|c|l|}
        \hline Objet & Symbole & Regex ou Formule \\
        \hline \hline Marqueur d'escroete & $mEs$ & /\textasciicircum[IV]+([1-9])?\$/ \\
        \hline Section d'escroete & $sEs$& $ sEs_{[n]} =  T[mEs_{[n]}:mEs_{[n+1]}-1] $\\
        \hline  Définition d'escroete & $dEs $& $ dEs_{[n]} = T[mEs_{[n]}:mCo_{[1]}-1] $\\
        \hline  Marqueur de connétablie & $mCo$ & /[0-9]+°/ \\
        \hline Section de connétablie & $sCo $& $sCo_{[n]} =  sEs[mCo_{[n]}:mCo_{[n+1]}-1] $\\
        \hline Définition de connétablie & $dCo$ & $ dCo_{[n]} = sEs[mCo_{[n]}:mRdV_{1]}-1] $\\
            & & $ dCo_{[n]} = sEs[mCo_{[n]}:mRe_{1]}-1] $ \\
        \hline  Marqueur de rang de voie & $mRdv$ & /\textasciicircum[AB]\$/ \\
        \hline Section de rang de voie & $sRdv$ &$ sRdv_{[n]} = sCo[mRdv_{[n]}:mRdv_{[n+1]}-1] $\\
        \hline Marqueur de rente & $mRe$ & /[0-9]\{2,\}./ \\
        \hline Section de rente & $sRe$& $sRe_{[n]} = sRdv[mRe_{[n]}:mRe_{[n+1]}-1]$ \\
            & &  $ sRe_{[n]} = sCo[mRe_{[n]}:mRe_{[n+1]}-1]$ \\
        \hline
    \end{tabular}
    \caption{Formules et expression régulières pour la segmentation du registre}
    \label{regexSeg}
\end{table}
\vspace{0,5cm}

\subsection{Évaluation}
\subsection{Correction}