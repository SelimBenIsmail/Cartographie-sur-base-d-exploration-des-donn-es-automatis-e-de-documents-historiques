\section{Extraction de l'information}
L'extraction se compose des phases de segmentation du texte, de la reconnaissance des entités nommées (REN), de l'analyse des relations entre entités. Toutes les étapes qui extraient, d'une manière ou d'une autre, de l'information depuis le document source ou d'un fragment de celui-ci. 
La finalité des opérations d'extraction de l'information dans le cadre de ce processus est de fournir un tableau de données, ou un système de table de données, reprenant toutes les informations semi-structurées du document source mises sous forme structurée, de sorte que ces données puissent directement servir à des études quantitatives.
\subsection{Segmentation du texte}
La segmentation du texte est l'une des premières phases du processus. Comme déjà expliqué, les rentes sont ordonnées dans le registre en fonction d'éléments topographiques (escroete, connétablie, rang de voie)

\subsection{Reconnaissance et extraction des entités nommées}

\subsection{Analyse des relations entre entités}