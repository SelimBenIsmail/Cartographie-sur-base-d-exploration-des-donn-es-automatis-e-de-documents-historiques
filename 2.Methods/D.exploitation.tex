\section{Exploitation de l'information}
Une fois, les données extraient et traitées, elles peuvent finalement être exploitées.
Les phases d'exploitation de l'information  arrivent à la fin du processus et vont permettre de matérialiser les objectifs de recherches. Ces phases d'exploitation des données comprennent la modélisation des graphes planaires à partir des informations extraites et traitées, et la transposition de ces graphes en cartographie.

\subsection{Théorie des graphes}
La théorie des graphes, que l'on nomme aussi fréquemment <<l'analyse de réseau>>, est une discipline des mathématiques qui consiste en l'étude des graphes et des réseaux qu'ils forment. Cette discipline utilise un vocabulaire qui lui est propre et dont la terminologie sera vue, de manière non exhaustive, dans la sous-section suivante.

\subsubsection{Terminologie et définitions}
Un graphe (\textit{G})  est défini comme un ensemble fini de sommets\footnote{Les sommets sont également fréquemment désignés par les appellations <<noeuds>> ou <<vertices>> (<<vertex>> au singulier) } (\textit{S}) et un ensemble fini, mais éventuellement vide, d'arêtes (\textit{A}) reliant des couples de sommets (\textit{s}). Plus formellement, un graphe peut être défini de la sorte : 
 \[ \boxed{ \; G = (S,A) \text{ tel que }A = (s_{n},s_{m}) \in S \; }\]  
Un graphe est caractérisé par son \textbf{ordre},  soit  le nombre de sommets et sa \textbf{taille}, soit le nombre d'arêtes.

%sousgraphe, composant, articulation et isthme
Un graphe peut être vu comme un ensemble de \textbf{sous-graphes}: \[ \boxed{G'=(S',A') \in G \text{ tel que } S'\subseteq S \text{ et } A'\subseteq A} \]  
Lorsque ces sous-graphes sont déconnectés de l'ensemble, ils sont appelés \textbf{composant}. Si la suppression d'un sommet \textit{s} augmente le nombre de composants du graphe, alors \textit{s} est un \textbf{point d'articulation}. Si c'est la suppression d'une arête \textit{a} qui augmente le nombre de composants du graphe, alors \textit{a} est appelé \textbf{isthme}.

%communauté
Lorsqu'un ensemble de sommets fortement liés entre eux et faiblement liés aux autres sommets, que ce soit par le  poids ou le nombre  des arêtes, ils forment une \textbf{communauté}, ou un \textit{cluster}.

%adjacence, incidence, isthme et et degré
Deux sommets  sont reliés par une ou plusieurs arêtes, ils sont dits \textbf{adjacents} ou \textbf{voisins}, tandis qu'un sommet, adjacent à aucun autre sommet, est dit \textbf{isolé}. L'\textbf{incidence} d'une arête à un sommet fait référence à la connexion entre une de ses extrémités et ce sommet. Le \textbf{degré} d'un sommet correspond au nombre d'arêtes incidentes à celui-ci \parencite{beauguitte_graphes_2010,solnon_theorie_nodate}.

%type de graphe
Il existe également de nombreux différents types de graphe se distinguant les uns des autres par une série d'opposition, en l'occurrence, dans la cadre de ce travail, il sera question uniquement de graphes planaires non orientés. C’est-à-dire que, d'une part, les arêtes n'ont pas de sens\footnote{Représenté par des flèches sur les schémas.}, elles ne caractérisent pas de liens de type <<origine/destination>>, et que, d'autre part, le graphe peut être projeté sur un plan sans que les arêtes se croisent entre elles. Il s'agit là du type de graphe privilégié pour la représentation de l'information topographique \parencite{beauguitte_graphes_2010}.

\subsection{Préparation des données}
%tableau
L'extraction des relations abordée précédemment dans la section \textit{Extraction de l'information}, a permit d'extraire les anthroponymes détectés dans le document source et de les mettre sous forme de couple dans un tableau de données  <<origine/destination>>.
Ce sont ces binômes dans le tableau de données qui définissent les extrémités des arêtes entre chaque sommet lors de la modélisation du graphe.

%nettoyage
Mais avant que la modélisation n'ait lieu, il est possible d'épurer le tableau de certaines entrées qui alourdissent le graphe de manière inutile. Parmi ces entrées, il y a les lignes du tableau où un même anthroponyme est à la fois l'origine et la destination ; ce qui se matérialise dans le graphe par une arête formant une boucle. Il y a également les lignes en doublons, mais aussi des lignes contenant les mêmes anthroponymes qu'une autre ligne, mais dans des colonnes différentes. Or, le graphe étant non orienté, le fait qu'un anthroponyme soit <<origine>> ou <<destination>> dans le tableau n'a aucune implication. Les deux anthroponymes d'une même ligne du tableau peuvent être transposés sans créer de problème et donc être potentiellement considérés comme des doublons avec d'autres lignes. 

\subsection{Modélisation des graphes}
La modélisation des graphes planaires est processus automatisé qui s'exécute depuis \textit{Rstudio}. Les graphes sont générés à l'aide de la bibliothèque \textit{Igraph}. Les fonctions fournies par la bibliothèque placent les sommets et les arêtes depuis le tableau de données <<origine/destination>> sur un plan, et suivant le paramétrage de la fonction, font en sorte qu'aucune d'elles ne se croise\footnote{Dans la pratique, une fois un nombre de sommets trop important, ou en cas de mauvais paramétrages, il est possible que les fonctions d'\textit{Igraph} croise des arêtes, bien qu'en théorie, ce n'est pas censé être la cas.}. 

\subsection{Transposition des graphes en cartographies}
Lorsque les graphes sont modélisés, afin d'arriver à une cartographie, il faut encore matérialiser l'information géographique en ajustant la disposition spatiale des sommets et des arêtes. Cette transposition des graphes en une cartographie constitue la dernière phase du système et la finalisation de l'objectif recherché. 

%étape 1 : attribution des informations topographique aux sommets
La première étape consiste donc à attacher les informations topographiques extraites lors de la segmentation sous forme d'attributs aux sommets du graphe. Cependant, un problème apparaît rapidement : les informations topographiques récoltées ne sont exactement relatives qu'au sujet de la rente ; soit l'anthroponyme détecté dans la première sous-chaîne. Les sommets modélisant les anthroponymes récoltés uniquement dans la seconde sous-chaîne ne portent donc aucune information topographique certaine. 

Pour pallier à cela, la règle suivante est définie : sachant, qu'à chacun de ces sommets est adjacent, au moins, un sommet dont  nous disposons d'informations sûres issues du tableau de données principales, deux cas de figure sont identifiables.
\begin{itemize}
    \item Soit, tous les sommets adjacents partagent des attributs ayant la même valeur, dans ce cas,  le sommet << incertain >> reçoit les attributs qui ont une valeur identique chez tous les sommets voisins << certain >>.
    \item Soit, les sommets voisins ont des attributs de valeurs différentes, dans ce cas, les attributs du sommet << incertain >>  restent sans valeur.
\end{itemize}
Ces sommets pourront ensuite être rassemblés par escroete ou connétablie.

%étape 2
Ensuite, il s'agit d'importer le fond de carte et d'y placer les repères topographiques identifiés dans le registre de rentes d'après leurs coordonnées géographiques. Ces repères comprennent les portes et enceintes de la ville, les connétablies citées dans le registre ainsi que des points de << concordances >> faisant le lien entre des emplacements de rentes et des éléments topographiques précis. Ces derniers permettront lors de l'étape suivante d'<< ancrer >> le graphe à des éléments topographiques dont la disposition spatiale est fixée sur la carte par des coordonnées géographiques.

% étape 3
La dernière étape devait consister à relier le graphe des rentes aux points précédemment établis. En reliant des sommets aux << points de concordances >>, il était supposé que le graphe se réoriente en fonction de ceux-ci. 

% étape 3 bis
Pour des raisons expliquées dans le chapitre suivant, cette étape n'a pas pu aboutir et la méthode a dû être revue  en conséquence.
Le tableau de données, précédemment créé, contenant les informations récoltées sur les sommets, a été récupéré pour être fusionné avec celui des connétablies. Dès lors, les anthroponymes ont pu être ancrés sur le fond de carte  par les connétablies auxquelles ils sont rattachés.
