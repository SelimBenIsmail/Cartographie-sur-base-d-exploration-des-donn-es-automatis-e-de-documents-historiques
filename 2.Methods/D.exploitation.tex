\section{Exploitation de l'information}
Une fois les données extraient et traitées, elles peuvent finalement être exploitées.
Les phases d'exploitation de l'information  arrivent à fin du processus et vont permettre de matérialiser les objectifs de recherches.

Ces phases d'exploitation des données comprennent la modélisation des graphes planaires à partir des informations extraites et traitées, et la transposition de ces graphes en cartographie.

\subsection{Théorie des graphes}
La théorie des graphes, que l'on nomme aussi fréquemment <<l'analyse de réseau>>, est une discipline des mathématiques qui consiste en l'étude des graphes et des réseaux qu'ils forment. Cette discipline utilise un vocabulaire qui lui est propre et dont la terminologie sera détaillée de manière non-exhaustive dans la sous-section suivante.

\subsubsection{Terminologie et définitions}
Un graphe (\textit{G})  est défini comme un ensemble fini de sommets (\textit{S}) et un ensemble fini, mais éventuellement vide, d'arêtes (\textit{A}) reliant des couples de sommets (\textit{s}). Plus formellement, un graphe peut être défini de la sorte : 
 \[ \boxed{ \; G = (S,A) \text{ tel que }A = (s_{n},s_{m}) \in S \; }\]  
Un graphe est caractérisé par son \textbf{ordre},  soit  le nombre de sommets et sa \textbf{taille}, soit le nombre d'arêtes.

%sousgraphe, composant, articulation et isthme
Un graphe peut être vu comme un ensemble de \textbf{sous-graphes}: \[ \boxed{G'=(S',A') \in G \text{ tel que } S'\subseteq S \text{ et } A'\subseteq A} \]  
Lorsque ces sous-graphes sont déconnectés de l'ensemble ils sont appelés \textbf{composant}. Si la suppression d'un sommet \textit{s} augmente le nombre de composants du graphe, alors \textit{s} est un \textbf{point d'articulation}. Si c'est la suppression d'une arête \textit{a} qui augmente le nombre de composant du graphe, alors \textit{a} est appelée \textbf{isthme} \parencite{beauguitte_graphes_2010,solnon_theorie_nodate}.

%communauté
Lorsqu'un ensemble de sommets est fortement liés entre eux et faiblement liés aux autres sommets, que ce soit par le  poids ou le nombre  des arêtes, ils forment une \textbf{communauté}, ou un \textit{cluster}.

%adjacence, incidence, isthme et et degré
Deux sommets reliés par une, ou plusieurs, arêtes sont dit \textbf{adjacents} ou \textbf{voisin}, tandis qu'un sommet adjacent à aucun autre sommet est dit \textbf{isolé}. L'\textbf{incidence} d'une arête à un sommet fait référence à la connexion entre une de ses extrémités et ce sommet. Le \textbf{degré} d'un sommet correspond au nombre d'arêtes incidentes à celui-ci \parencite{beauguitte_graphes_2010,solnon_theorie_nodate}.

%type de graphe
Il existe également de nombreux différents types de graphe se distinguant les uns des autres par une série d'opposition, en l'occurrence, dans la cadre de ce travail, il sera question uniquement de graphes planaire non orienté. c'est à dire que, d'une part, les arêtes n'ont pas de sens, elle ne caractérisent pas de liens <<origine/destination>>, et que, d'autre part, le graphe peut être projeté sur un plan sans que les arêtes ne se croisent entre-elles. Il s'agit là du type de graphe privilégié pour la représentation de l'information topographique \parencite{beauguitte_graphes_2010}.

\subsection{préparation des données}
%tableau
L'extraction des relations abordée précédemment dans la section \textit{Extraction de l'information}, a permit d'extraire les anthroponymes détectés dans le document source et de les mettre sous forme de couple dans un tableau de données  <<origine/destination>>.
Ce sont ces bînomes dans le tableau de données qui définissent les extrémités des arêtes entre chaque sommet lors de la modélisation du graphe.

%nettoyage
Mais avant que la modélisation n'ait lieu, il est possible d'épurer le tableau de certaines entrées qui alourdissent le graphes de manière inutile. Parmi ces entrés, il y a les lignes du tableau où un même anthroponyme est à la fois origine et destination et qui se matérialise dans le graphe par un arête formant une boucle. Il y aussi les lignes doublons, mais également les lignes contenant les mêmes anthroponymes qu'une autre ligne, mais à des colonnes différentes. Or, le graphe étant non-orinté, le fait qu'un anthroponyme soit <<origine>> ou <<destination>> dans le tableau, n'a aucune implication. Les deux anthroponymes d'une même ligne du tableau peuvent être transposés sans aucun problème et donc être éventuellement considérés comme des doublons avec d'autres lignes. 

\subsubsection{Application aux données}
La modélisation des graphes planaires est processus automatisé qui s'exécute depuis \textit{Rstudio}. Les graphes sont générés à l'aide de la bibliothèque \textit{Igraph}. Les fonctions fournies par la bibliothèque placent les sommets et les arêtes depuis le tableau de données <<origine/destination>>, et, suivant le paramétrage encodé, fait en sorte qu'aucune d'elles se croisent\footnote{Dans la pratique, une fois un nombre de sommets trop important, ou en cas de mauvais paramétrages, il est possible que les fonctions d'\textit{Igraph} croise des arêtes, bien qu'en théorie, ce n'est pas censé être la cas.}. 

\subsection{Transposition des graphes en cartographie}
%draft
Une fois les graphes modélisés, il faut encore matérialiser l'information géographique  en  ajustant la disposition spatial des sommets et des arêtes pour arriver à une cartographie.
Cette transposition des graphes en une cartographie constitue la dernière étape du système et la finalisation de l'objectif recherché. Pour arriver à cette fin, deux stratégies, pouvant se combiner, sont étudiées.

%stratégie
Les deux approches partent de la transformation du graphe en un nouveau graphe multiparti --- un graphe comprenant plusieurs types de sommets différents --- et l'ajout d'une série de nouveaux sommets reliés aux précédents. Ensuite, la première stratégie consiste à rajouter de nouveaux sommets de type <<connétablie>> et de type <<escroete>> à relier aux sommets de type << rente>> qui y sont géographiquement incluses,  de sorte à créer des communautés. Un \textit{layout} rapprochant les sommets d'une même communauté entre eux serait alors utilisé. La seconde stratégie, quant à elle, propose la création manuelle d'une liste de sommets correspondant aux croisements des connétablies. Ces sommet prendraient la valeur des coordonnées géographiques précises des intersections qu'ils représentent, et les nouvelles arêtes dessinerait alors les sections de rues entre ces croisements. En reliant quelques sommets du graphe initial au nouveau, dont la disposition spatiale des sommets est fixées par des coordonnées géographique, le graphe initiale devraient se réorienté en fonction de ce second graphe.