\chapter{Introduction}
\section{Introduction générale}
 Ce mémoire détaille la conception d’un système dont l’objectif est de cartographier la ville de Douai à l’époque du XIIIe siècle sur la base de fouille de texte automatisée à travers un registre de rentes d'époque, celui de Jehan de France, un riche marchand drapier et patricien de la ville de Douai. Trois thèmes majeurs des sciences de l'information et de la communication sont au centre de cette recherche :
 
 La première thématique est l’application des outils de traitement automatique des langues naturelles (TALN) aux textes historiques. L’émergence des humanités numériques a ouvert la recherche en sciences humaines à de nouvelles méthodes d’analyses qui ont, depuis, déjà grandement prouvé leur efficacité. Cependant, dans le cadre de l’étude de textes historique, les outils utilisés par celles-ci se heurtent à de lourdes complications. L’une des principales est le langage dans lequel sont écrits des textes historiques : les outils de fouilles de textes sont généralement conçus pour traiter les langues très représentées telles que l’anglais ou le français et non des langues anciennes. Pour cause, une grande partie de ces outils ont besoin de corpus de textes conséquent ou de dictionnaires pour être performants. Un facteur accentuant cet état de fait est qu’il est relativement compliqué de composer un corpus de textes historiques : d’une part, parce que les textes anciens, étant sur des supports physiques, doivent être numérisés et ocrisés avant de pouvoir être utilisés, et d’autre part, car les sources écrites sont régulièrement lacunaires ou amputées d’une partie de leur contenu. La personne cherchant à traiter de façon automatisée des textes historiques devra donc se montrer créative.
 
La seconde thématique est l’application de la théorie des graphes au domaine de l’Histoire. La théorie des graphes trouve ses premières origines dans les travaux du mathématicien Leonhard Euler en 1735 et son problème des « sept ponts de Königsberg »\parencite{euler_solutio_1735}. Les graphes permettent de modéliser des problèmes complexes sous la forme d’une structure de sommets reliés entre eux par des arêtes où les sommets renvoient à des référents — cela peut être des personnes, des lieux, des propositions, etc. — et les arêtes aux relations qui les relient entre eux. Si la théorie fut d’abord utilisée pour résoudre des problèmes de géométrie ou de mathématique, elle s’est popularisée dans le domaine des sciences humaines dans les années 1960 afin de modéliser diverses problématiques.

Finalement, la troisième thématique est celle de la visualisation des données sous forme de cartographies. L’intérêt de la cartographie dans la recherche en Histoire est multiple. Le premier est que ces cartes sont la représentation d’un grand nombre de données agrégées. Elles permettent donc d’analyser des phénomènes déjà connus d’un point de vue plus éloigné et plus global, permettant ainsi de découvrir de nouvelles relations corollaires ou causales entre différents évènements qui ne semblaient pas forcément liés aux premiers abords. À l’instar de l’exploration de graphiques et données statistiques, l’exploration des différents types de cartes se pose comme une approche complémentaire à l’étude des sources primaires dans les recherches en Histoire.

Un autre intérêt majeur de la visualisation des données se trouve dans la communication et la transmission des savoirs. En effet, l’information visuelle, pour peu qu’elle soit correctement organisée et présentée, s’appréhende et s’assimile avec plus d’aisance que l’information textuelle brute. Or c’est dans cette transmission que l’étude de l’Histoire prend son plein sens.

\section{Question de recherche}
 La question de recherche est \emph{\og La théorie des graphes, appliquée au domaine des patrimoines immobiliers en Flandre médiévale, couplée à une fouille de texte automatisée dans des textes historiques, permet-elle de produire des visuels cartographiques pertinents à l’étude de l’histoire urbaine ?\fg{}} 
 
 Derrière cette question de recherche relativement vaste, se dissimule en réalité l'évaluation du résultat de trois objectifs distincts mis bout à bout. À savoir : développer des algorithmes de fouille et traitement de textes adaptés à un document historique, appliquer la théorie des réseaux aux résultats obtenus par cette fouille afin de représenter des liens de mitoyennetés, et finalement, transposer ces graphes en un visuel cartographique afin d'ouvrir de nouvelles voies de recherche dans l'étude du développement urbain.
 
\section{Hypothèse}
Le mémoire présente un travail de recherche et la conception d'un prototype, d'un \textit{proof of concept} permettant de répondre à la question de recherche. Très peu de travaux, si ce n'est aucun, n'ont été publiés sur cette méthode d'analyse des documents médiévaux. Il est donc difficile d'émettre des hypothèses quant aux limites qui seront rencontrées et sur les résultats qui seront obtenus.

Toutefois, cette méthode semble réalisable dans la théorie, et dans une optique optimiste,  les visuels produits devraient permettre de rendre compte des propriétés immobilières de Jehan de France, de la disparité et des types des loyers en fonction des quartiers de la ville, de la dispersion des noms de famille, etc. Des éléments qui, croisés  à d'autres sources, pourraient valider ou invalider d'autres hypothèses. 

