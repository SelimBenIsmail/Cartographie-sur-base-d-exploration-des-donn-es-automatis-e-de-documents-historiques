\chapter{Introduction}
\section{Introduction générale}
%Il y sera question de présenter les enjeux des thématiques abordées ainsi que les objectifs recherchés à travers ce mémoire%
 Ce mémoire détaillera la conception d’un système dont l’objectif est de cartographier la ville de Douai à l’époque du XIIIe siècle sur base de fouilles de texte automatisées à travers un registre de rentes d'époque, celui de Jehan de France, un riche marchand drapier et patricien de la ville de Douai. Trois thématiques majeurs des sciences de l'information et de la communication seront au centre de cette recherche :
 
 La première thématique est l’application des outils de traitement automatique des langues naturelles (TALN) aux textes historiques. L’émergence des humanités numériques a ouvert la recherche en sciences humaines à de nouvelles méthodes d’analyses qui ont, depuis, déjà grandement prouvé leur efficacité. Cependant, dans le cadre de l’étude de textes historique les outils utilisés par celles-ci se heurtent à de lourdes complications. L’une des principales est le langage dans lequel sont écrits des textes historiques : les outils de fouilles de textes sont généralement conçus pour traiter les langues très représentées telles que l’anglais ou le français et non des langues anciennes. Pour cause, une grande partie de ces outils ont besoin de corpus de textes conséquent ou de dictionnaires pour être performants. Un facteur accentuant cet état de fait est qu’il est relativement compliqué de composer un corpus de textes historiques : d’une part, car les textes anciens, étant sur des supports physiques, doivent être numérisés et ocrisés avant de pouvoir être utilisés, et d’autre part, car les sources écrites sont régulièrement lacunaires ou amputées d’une partie de leur contenu. La personne cherchant à traiter de façon automatisée des textes historiques devra donc se montrer créative.
 
La seconde thématique est l’application de la théorie des graphes au domaine de l’Histoire. La théorie des graphes trouve ses premières origines dans les travaux du mathématicien Leonhard Euler en 1735 et son problème des « sept ponts de Königsberg ». Les graphes permettent de modéliser des problèmes complexes sous la forme d’une structure de sommets reliés entre eux par des arêtes où les sommets renvoient à des référents — cela peut être des personnes, des lieux, des propositions, etc. — et les arêtes aux relations qui les relient entre eux. Si la théorie fut d’abord utilisée pour résoudre des problèmes de géométrie ou de mathématique, elle s’est popularisée dans le domaine des sciences humaines dans les années 1960 afin de modéliser diverses problématiques.

Finalement, la troisième thématique est celle de la visualisation des données sous forme de cartographies. L’intérêt de la cartographie dans la recherche en Histoire est multiple. Le premier est que ces cartes sont la représentation d’un grand nombre de données agrégées. Elles permettent donc d’analyser des phénomènes déjà connus d’un point de vue plus éloigné et plus global, permettant ainsi de découvrir de nouvelles relations corollaires ou causales entre différents évènements qui ne semblaient pas forcément liés aux premiers abords. À l’instar de l’exploration de graphiques et données statistiques, l’exploration des différents types de cartes se pose comme une approche complémentaire à l’étude des sources primaires dans les recherches en Histoire.

Un autre intérêt majeur de la visualisation des données se trouve dans la communication et la transmission des savoirs. En effet, l’information visuelle, pour peu qu’elle soit correctement organisée et présentée, s’appréhende et s’assimile avec plus d’aisance que l’information textuelle brute. Or c’est dans cette transmission que l’étude de l’Histoire prend son plein sens.

\section{Question de recherche}
 la question de rechercher est "La théorie des réseaux, appliquée au domaine des patrimoines immobiliers, couplée à une fouille de texte automatisée dans des texte historique, permet-elle de produire des visuels cartographiques pertinents pour l’étude de l’histoire urbaine ?"
 
 Derrière cette question de recherche très vaste, se cache en réalité l'évaluation du résultat de trois objectif distincts mis bout à bout : développer des algorithmes de fouille et traitement de texte à un document historique, appliquer la théorie des réseaux aux résultats de cette fouille et finalement, transposer ces graphes en un visuel cartographie afin d'ouvrir de nouvelles voies de recherche dans l'étude du développement urbain.
\section{Hypothèse}
%Cette partie présentera les hypothèses émises pour répondre à la question de recherche.%
Le mémoire présente un travail de recherche et la conception d'un proof of concept. Très peu de travaux, si ce n'est aucun, n'ont été publiés sur cette méthode. Il est donc  difficile d'émettre des hypothèses, avec certitude,  sur les limites que nous rencontrerons et sur les résultats que nous obtiendrons avec cette méthode.

toutefois, cette méthode semble réalisable sur le papier, rien n'indique le contraire. Et dans une optique optimiste,  les visuels produit devrait permettre de rendre compte des propriétés immobilières de Jehan de France, de la disparité et du types des loyers à travers  la ville, de la dispersion des noms de familles, etc.
Tant d'éléments qui pourrait valider ou invalider tant d'autres hypothèses. 

\section{Contexte historique}
\subsection{La ville de Douai au XIIIe siècle}

Douai trouve ses origines au sixième siècle sur un îlot de la Scarpe, une rivière traversant le nord de la France et se déversant dans le fleuve de l'Escaut. Les terres fertiles des alentours offrent une rapide croissance à ce qui n'est alors encore qu'une petite agglomération rurale. Rapidement l'agglomération s'étend de part en part de la Scarpe et s'équipe d'une place-forte ainsi que d'une première muraille. 

Forte de sa culture de blé et de ses accès fluviaux, la ville gagna en richesse et devient un point de commerce important de la région, ce qui attira marchands et artisans. La ville s'organise alors en quartiers et corporations d'artisans. On retrouve par exemple la rue des ferronniers, des foulons ou encore des tanneurs. Les rues prennent à cette époque une dénomination en fonction de leurs usages.

Si le commerce céréalier fit démarrer la machine économique, au XIIe et XIIIe siècle, Douai se forge une réputation autour du commerce de draperie. 
Grâce aux foires de Champagne, la draperie Douaisienne s'exporte et gagne  une renommée à travers à toute l'Europe. Rapidement la production de laine locale ne suffit plus à combler la demande en étoffe et un fructueux commerce s'installe avec Angleterre qui fournit alors Douai en laine.

Au fil de l'Histoire, Douai ne cessa d'être tantôt Française, tantôt Flamande. Cependant, la richesse et la puissance de la ville, fît  qu'elle garda toujours une certaine indépendance vis à vis des rois de France ou des Compte de Flandre.

En XXXX, un premier échevinage apparaît, constitué de 12 riches notables douaisiens. Ceux-ci régissent les activités de la ville au moyens de bans, une forme d'arrêté législatif. Ceci témoigne de la place de plus importante qu'occupe les marchands dans la vie de la ville et fur et à mesure que l'économie de la ville continue de se développer.

Parallèlement, une nouvelle forme de capitalisme émerge dans la cité : l'hiterage, le commerce de biens fonciers. Ces marchands notables acquièrent les droits sur les biens immobiliers et jouissaient des rentes de ces derniers.


\subsection{Sire Jehan de France}
%Il s’agira d’une brève présentation de la vie de Jean de France et explications sur l’acquisition de ses biens fonciers.%
\subsection{Registre de rentes des propriétés foncières de Jehan de France}
%Cette partie traitera de l’origine du carnet de rentes, de la façon dont sont agencées les informations et des données qu’il contient et que l’on cherche à extraire.%

