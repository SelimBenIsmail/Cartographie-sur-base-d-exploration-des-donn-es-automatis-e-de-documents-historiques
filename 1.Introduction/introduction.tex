\chapter{Introduction}
\section{Introduction générale}
%Il y sera question de présenter les enjeux des thématiques abordées ainsi que les objectifs recherchés à travers ce mémoire%
\section{Question de recherche}
%La théorie des réseaux, appliquée au domaine des patrimoines immobiliers,permet-elle de produire des visuels cartographiques pertinents pour l’étude de l’histoire urbaine ?%
\section{Hypothèse}
%Cette partie présentera les hypothèses émises pour répondre à la question de recherche.%
\section{Contexte historique}
\subsection{La ville de Douai au XIIIe siècle}
%Cette partie consistera en une description du contexte administratif et économique de la ville et de son rôle dans l’histoire de Flandre.
%
\subsection{Sire Jehan de France}
%Il s’agira d’une brève présentation de la vie de Jean de France et explications sur l’acquisition de ses biens fonciers.%
\subsection{Registre de rentes des propriétés foncières de Jehan de France}
%Cette partie traitera de l’origine du carnet de rentes, de la façon dont sont agencées les informations et des données qu’il contient et que l’on cherche à extraire.%

